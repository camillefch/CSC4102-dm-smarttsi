\documentclass[11pt,article]{article}
\usepackage[utf8]{inputenc}
\usepackage[T1]{fontenc} % caractères accentués en entrée, dans emacs
\usepackage[french]{babel}
\FrenchFootnotes
\selectlanguage{french}
\usepackage{a4wide} % possibilité d'utiliser toute la page a4
% selon GUT#33, avril 2007, page 13, empagement
% largeur des textes (ou justification) = 15cm
% hauteur du rectangle d'empagement = 23cm
% blanc de couture = 2/5 (21-15) = 2.4 = inner = right
% blanc de grand fond = 3/5 (21-15) = outer = left
% blanc de tête = 2/5 (29,7-23) = top
% blanc de pied = 3/5 (29,7-23) = bottom
%\usepackage[a4paper,twoside=true,right=2.4cm,left=3.6cm,top=2.68cm,bottom=4.02cm]{geometry} 
% selon CFSE 2006
% - largeur des textes (ou justification) : 16cm (2cm de marge, et 1cm
%   de reliure) ;
% - hauteur des textes, y compris les notes : 23cm (2,5cm de marge
%   haute et 2cm de marge basse) ; 1ère page de : 36pts
%   d'espacement avant le titre ;
\oddsidemargin   -4mm           % 3cm a gauche des impaires
\evensidemargin   4mm           % 2cm a gauche des paires
\topmargin       -18mm          % 2.5cm en haut
\headheight       13mm          % taille de l'entete (lignes)
\headsep          24pt          % espace entre entete et texte
\footskip         30pt          % espace entre pied de page et texte
\textheight      230mm          % longeur du texte
\textwidth       160mm          % largeur du texte
\parskip 1pt                    % pas de sauts entre paragraphes
%\parindent 0pt                  % largeur de l'indentation
\usepackage{graphicx} % figure postcript avec latex,
		      % figure png avec pdflatex, au lieu d'utiliser epsfig
\usepackage[usenames,dvipsnames,table]{xcolor}
\usepackage{paralist}
\usepackage{ifthen}
\usepackage{amssymb}
\usepackage{amsfonts}
\usepackage{amsmath}
\usepackage{eurosym}
\usepackage{textcomp}
\usepackage{listings}
\lstset{language=Java,numbers=left,numberstyle=\tiny,stepnumber=4,numbersep=5pt,xleftmargin=5pt}

\usepackage{alltt}
\usepackage{longtable}

% adjust word spacing less strictly
% as result, some spaces between words may be a bit too large,
% but long words will be placed properly.
\sloppy

\newcommand{\cmt}[1]{\texttt{<}\textbf{--~#1~--}\texttt{>}}

\usepackage{lineno}
\usepackage{xspace}

\setlength{\marginparwidth}{1cm}
\setlength{\marginparsep}{10pt}
\reversemarginpar
\newcounter{usecasehaute}
\newcommand{\haute}{Haute}
\newcommand{\moyenne}{Moyenne}
\newcommand{\basse}{basse}
\newcommand{\usecase}[4]{\item \marginpar{\vspace{5pt}\ifthenelse{\equal{#1}{Haute}}{\centering\textsc{#1}\stepcounter{usecasehaute}\newline n$^{\circ}$ \theusecasehaute}{\ifthenelse{\equal{#1}{Moyenne}}{#1}{\small #1}}} #2 \begin{itemize}\item précondition~: #3 \item postcondition~: #4\end{itemize}}
\newcommand{\priorityusecase}[2]{\item \marginpar{\vspace{5pt}\ifthenelse{\equal{#1}{Haute}}{\centering\textsc{#1}\stepcounter{usecasehaute}\newline n$^{\circ}$ \theusecasehaute}{\ifthenelse{\equal{#1}{Moyenne}}{#1}{\small #1}}} #2}
\newcommand{\casusecase}[4]{\usecase{#1}{#2}{#3}{#4}}

\newcommand{\nullvalue}{\textsf{null}\xspace}
\newcommand{\emptyvalue}{\ensuremath\mathrm{vide}}
\newcommand{\invariant}{\ensuremath\mathrm{invariant}}

\begin{document}
\title{Projet CSC4102: \textit{Smart} Traçabilité pour Suivi des Incidents}
\author{Nom Prénom Étudiant1 et Nom Prénom Étudiants2}
\date{Année 2021--2022~---~\today}
\maketitle

\vfill

\tableofcontents

\newpage

\section{Spécification}

\subsection{Diagrammes de cas d'utilisation}

{\noindent\color{red}\textbf{Le diagramme suivant est à compléter. Ce
    commentaire est à retirer ensuite.}}

\begin{figure}[!ht]
\begin{center}
\includegraphics[scale=0.6]{Diagrammes/smarttsi_uml_diag_cas_utilisation}
\caption{Diagramme de cas d'utilisation.}
\end{center}
\label{usecase_modelio}
\end{figure}

\newpage

\subsection{Priorités, et préconditions et postconditions des cas d'utilisation}

Les priorités des cas d'utilisation pour le sprint~1 sont choisies
avec les règles de bon sens suivantes:
\begin{compactitem}
\item pour retirer une entité du système, elle doit y être. La
priorité de l'ajout est donc supérieure ou égale à la priorité du
retrait;
\item pour lister les entités d'un type donné, elles doivent y être. La
priorité de l'ajout est donc supérieure ou égale à la priorité du
listage;
\item il est \textit{a priori} possible, c.-à-d. sans raison
contraire, de démontrer la mise en œuvre d'un sous-ensemble des
fonctionnalités du système, et plus particulièrement la prise en
compte des principales règles de gestion, sans les retraits ou les
listages.
\item la possibilité de lister aide au déverminage de l'application
pendant les activités d'exécution des tests de validation.
\end{compactitem}
Par conséquent, les cas d'utilisation d'ajout sont \textit{a priori}
de priorité <<~haute~>>, ceux de listage de priorité <<~moyenne~>>, et
ceux de retrait de priorité <<~basse~>>.

\bigskip

Dans la suite, nous donnons les préconditions et postconditions pour
les cas d'utilisation de priorité <<~\haute~>>. Pour les autres, nous
indiquons uniquement leur niveau de priorité.

\bigskip

{\noindent\color{red}\textbf{La liste de préconditions et
    postconditions suivante est à modifier et compléter. Ce
    commentaire est à retirer ensuite.}}

\bigskip

Voici les préconditions et postconditions des cas d'utilisation du
premier sprint:
\begin{compactitem}
\usecase{\haute}{Déposer un envoi {\color{red}(à compléter)}, sans adjoindre une source de données IoT}
        %% précondition
        {\newline
          $\land$ identifiant de l'envoi bien formé (non \nullvalue
          et non vide)
          \newline
          $\land$ origine bien formée (non \nullvalue
          et non vide)
          \newline
          $\land$ destination bien formée (non \nullvalue
          et non vide)
          \newline
          $\land$ envoi avec cet identifiant inexistant
          }
        %% postcondition
        {\newline
          $\land$ envoi avec cet identifiant existant
          \newline
          $\land$ instant de dépôt est aujourd'hui
        }

\smallskip

\priorityusecase{\moyenne}{Lister les envois}
 
\smallskip

\priorityusecase{\basse}{retirer un envoi}
 
\end{compactitem}

\newpage

\section{Préparation des tests de validation}

\subsection{Tables de décision des tests de validation}

La fiche programme du module CSC4102 ne permettant pas de développer
des tests de validation couvrant l'ensemble des cas d'utilisation de
l'application, les cas d'utilisation choisis sont de
priorité \textsc{Haute}.

\bigskip

{\noindent\color{red}\textbf{La section est à compléter avec les
    tables de décision d'autres cas d'utilisation. Ce commentaire est
    à retirer ensuite.}}

\begin{table}[htbp!]
\begin{tabular}{|p{0.6\linewidth}|c|c|c|c|c|}
\hline
Numéro de test
&1&2&3&4&5\\
\hline
\hline
Identifiant de l'envoi bien formé (non \nullvalue et non vide)
&F&T&T&T&T\\
\hline
Origine bien formé (non \nullvalue et non vide)
& &F&T&T&T\\
\hline
Destination bien formée (non \nullvalue et non vide)
& & &F&T&T\\
\hline
\hline
Envoi avec cet identifiant inexistant
& & & &F&T\\
\hline
\hline
Création acceptée
&F&F&F&F&T\\
\hline
Instant de dépôt est aujourd'hui
&F&F&F&F&T\\
\hline
\hline
Nombre de jeux de test
&2&2&2&1&1\\
\hline
\end{tabular}
\caption{Cas d'utilisation <<~déposer un envoi {\color{red}(à compléter)}, sans adjoindre une source de données IoT~>>}
\end{table}

\newpage

\section{Conception}

\subsection{Diagramme de classes}

{\noindent\color{red}\textbf{Le diagramme de classes suivant est à
    compléter. Ce commentaire est à retirer ensuite.}}

\begin{figure}[!ht]
\begin{center}
\includegraphics[scale=0.6]{Diagrammes/smarttsi_uml_diag_classes}
\caption{Diagramme de classes.}
\end{center}
\label{umlet_diag_classes}
\end{figure}

\newpage

\subsection{Diagrammes de séquence}

{\noindent\color{red}\textbf{La section est à compléter avec les
    diagrammes de séquence de vos cas d'utilisation les plus
    importants, c'est-à-dire avec ceux de priorité haute. Ce
    commentaire est à retirer ensuite.}}

\begin{figure}[ht!]
\begin{center}
\includegraphics[scale=0.5]{Diagrammes/smarttsi_uml_diag_seq_deposer_envoi}
\caption{Diagramme de séquence du cas d'utilisation <<~déposer un envoi~>> {\color{red}(à compléter)}.}
\end{center}
\label{umlet_diag_sequence_deposer_envoi}
\end{figure}

\newpage

\section{Diagrammes de machine à états et invariants}

{\noindent\color{red}\textbf{La section est à compléter avec les
    diagrammes de machine à états et les invariants de vos classes les
    plus importantes. Ce commentaire est à retirer ensuite.}}

\subsection{Classes \textsf{Envoi}}

\begin{figure}[ht!]
\begin{center}
\includegraphics[scale=0.5]{Diagrammes/smarttsi_uml_diag_machine_a_etats_envoi}
\caption{Diagramme de machine à états de la classe \texttt{Envoi} {\color{red}(à compléter)}.}
\end{center}
\label{umlet_diag_machine_a_etats_envoi}
\end{figure}

\newcommand{\emptystring}{\ensuremath\mathrm{vide}}
\newcommand{\identifiant}{\ensuremath\mathrm{identifiant}}
\newcommand{\origine}{\ensuremath\mathrm{origine}}
\newcommand{\instantDepot}{\ensuremath\mathrm{instantDepot}}
\newcommand{\destination}{\ensuremath\mathrm{destination}}
\newcommand{\instantLivraison}{\ensuremath\mathrm{instantLivraison}}
\newcommand{\etat}{\ensuremath\mathrm{etat}}
L'invariant de la classe \textsf{Envoi} est le suivant {\color{red}(à compléter)}:
\begin{tabbing}
M \= M \= M \= M \= M \= M \= M \kill
\> $\land$ \> $\identifiant \neq \nullvalue \land \identifiant \neq \emptystring$\\
\> $\land$ \> $\origine \neq \nullvalue \land \origine \neq \emptystring$\\
\> $\land$ \> $\instantDepot \neq \nullvalue$\\
\> $\land$ \> $\destination \neq \nullvalue \land \destination \neq \emptystring$\\
\> $\land$ \> $\etat \neq \nullvalue$\\
\end{tabbing}

\newpage

\section{Préparation des tests unitaires}

{\noindent\color{red}\textbf{La section est à compléter. Ce
    commentaire est à retirer ensuite.}}

\subsection{Classe \textsf{Envoi}}

\begin{table}[!ht]
\begin{center}
\begin{tabular}{|p{0.5\linewidth}|c|c|c|c|c|c|}
\hline
Numéro de test
&1&2&3&4&5&6\\
\hline
\hline
\texttt{identifiant} $\neq \nullvalue \land \neg \emptystring$
&F&T&T&T&T&T\\
\hline
\texttt{origine} $\neq \nullvalue \land \neg \emptystring$
& &F&T&T&T&T\\
\hline
\texttt{instantDépôt} $\neq \nullvalue$
& & &F&T&T&T\\
\hline
\texttt{instantDépôt} n'est pas dans le futur
& & & &F&T&T\\
\hline
\texttt{destination} $\neq \nullvalue \land \neg \emptystring$
& & & & &F&T\\
\hline
\hline
$\identifiant' = \identifiant$
& & & & & &T\\
\hline
$\origine' = \origine$
& & & & & &T\\
\hline
$\instantDepot' = \instantDepot$
& & & & & &T\\
\hline
$\destination' = \destination$
& & & & & &T\\
\hline
$\instantLivraison' = \instantLivraison$
& & & & & &T\\
\hline
$\invariant$
& & & & & &T\\
\hline
Levée d'une exception&\textsc{oui}&\textsc{oui}&\textsc{oui}&\textsc{oui}&\textsc{oui}&\textsc{non}\\
\hline
\hline
Objet créé
&F&F&F&F&F&T\\
\hline
\hline
Nombre de jeux de test 
&2&2&1&1&1&1\\
\hline
\end{tabular}
\caption{Méthode \texttt{constructeurEnvoi} de la classe
  \texttt{Envoi} {\color{red}(à compléter)}}
\end{center}
\end{table}

\end{document}
